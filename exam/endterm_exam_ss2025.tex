% !TEX root = endterm_exam_ss2025.tex
% !TeX encoding = UTF-8
% !TeX spellcheck = en_US
\documentclass{article}
%%%%%%%%%%%%%%%%%%%%%%%%%%%%%%%%%%%%%%%%%%%%%%%%%%%%%%%%%%%%%%%%%%%%%%%%%%%%%%%%%%%%%%%%%%%%%%%%%%%%%%%%%%%%%%%%%%%%%%%%%%%%%%%%%%%%%%%%%%%%%%%%%%%%%%%%%%%%%%%%%%%%%%%%%%%%%%%%%%%%%%%%%%%%%%%%%%%%%%%%%%%%%%%%%%%%%%%%%%%%%%%%%%%%%%%%%%%%%%%%%%%%%%%%%%%%
\usepackage[a4paper,top=2cm,bottom=2cm,left=2.5cm,right=2.5cm]{geometry}
\usepackage{amssymb,amsmath,amsfonts}
\usepackage[english]{babel}
\usepackage[a4paper]{geometry}
\usepackage{enumitem}
\usepackage{booktabs}
\usepackage{csquotes}
\usepackage{url}
\usepackage{graphicx}
\usepackage{booktabs,longtable}
\usepackage{xcolor}
%\parindent0mm
%\parskip1.5ex plus0.5ex minus0.5ex
\numberwithin{equation}{section}
\renewcommand{\theequation}{\thesection.\arabic{equation}}

\begin{document}

\title{Computational Macroeconomics}
\author{Endterm Exam}
\date{Summer 2025}
\maketitle


\begin{itemize}
\item
Answer \textbf{all} of the following \textbf{three} exercises in English.
\item
All assignments will be given the same weight in the final grade.
\item
Hand in your solutions before Sunday August, 08 2025 at 20:00 (8pm).
\item
The solution files should contain your executable (and commented) Dynare and MATLAB files
  as well as all additional documentation as \texttt{pdf}, not \texttt{doc} or \texttt{docx}.
Your \texttt{pdf} files may also include scans or pictures of handwritten notes.
\item
Please e-mail ALL the solution files to \url{willi.mutschler@uni-tuebingen.de}.
\item
All students must work on their own, please also give your student ID number.
\item
It is advised to regularly check Mattermost and your emails in case of urgent updates.
\item
All papers and technical appendices are available on Mattermost.
All codes are available on GitHub.
\item Use of Large Language Models (LLMs): The use of AI-based language models (e.g., ChatGPT, Claude, Copilot, Grok, Mistral, etc.) is permitted for this exam under the following conditions:
  \begin{itemize}
    \item You must properly cite any LLM used for each question,
      including: (i) the specific model and version and (ii) a clear but concise description of how it was used and what tasks it assited with.
    Example of proper citation: ``ChatGPT o4-mini-high was used to verify LaTeX syntax for equation formatting and to proofread the English language.''\footnote{%
    Hint: You can prompt your LLM at the end of a chat session to provide a concise summary of the specific tasks.}
    Failure to properly cite LLM usage or using LLMs to generate substantive content will be considered academic misconduct.
    
    \item Acceptable uses include:
      \begin{itemize}
        \item Language correction and proofreading
        \item LaTeX, Markdown, MATLAB, Dynare, etc.\ syntax and coding assistance
        \item Clarification of mathematical notation and derivations
        \item Verification of standard concepts and definitions
      \end{itemize}
    \item The following uses are \textbf{not permitted}:
      \begin{itemize}
        \item Having the LLM solve the problems for you
        \item Copying explanations or derivations without understanding
        \item Using LLM-generated economic interpretations without critical evaluation
      \end{itemize}
    \item All mathematical derivations, economic interpretations, and problem solutions must be your own work and demonstrate your understanding of the material
    
  \end{itemize}
\item
If there are any questions, do not hesitate to contact Willi Mutschler.
\end{itemize}

\section*{Changelog}
\begin{itemize}
\item Version 1.0: First version of the exam.
\end{itemize}

\newpage

\section[Technology shocks with non-symmetric innovations]{Technology shocks with non-symmetric innovations\label{ex:Andreasen_2012}}
The goal of this exercise is to look into the third-order perturbation approach and replicate columns (2) and (3) of Table 2 in Andreasen (2012):
  \emph{On the effects of rare disasters and uncertainty shocks for risk premia in non-linear DSGE models},
  published in Review of Economic Dynamics (15).
For the following exercise, we will focus on the sections about rare disasters only.

\begin{enumerate}
\item
What is the main research question of the paper?
Why is the proposed third-order perturbation solution method appropriate to tackle this research question?

\item
Compared to the standard New Keynesian DSGE model we learned about in the lecture,
  what features are added in the paper and why? Do you find the modelling approach of a rare disaster convincing?

\item
The third-order perturbation solution in Dynare's model framework is given by:
\begin{align*}
y_t &= \bar{y}\\
&
\colorbox{gray!10}{\(%
+g_x (x_{t-1} - \bar{x}) + g_u u_{t}
\)}
\\
&
\qquad
\colorbox{gray!10}{\(%
+ g_{\sigma}
\)}
\\
&
\colorbox{gray!25}{\(%
+ \frac{1}{2} g_{xx} \left((x_{t-1} - \bar{x}) \otimes (x_{t-1} - \bar{x})\right)
  +           g_{xu} \left((x_{t-1} - \bar{x}) \otimes u_t\right)
+ \frac{1}{2} g_{uu} \left(u_t \otimes u_t\right)
\)}
\\
&
\qquad
\colorbox{gray!25}{\(%
+ g_{x\sigma} (x_{t-1} - \bar{x})
+ g_{u\sigma} u_t
\)}
\\
&
\qquad\qquad
\colorbox{gray!25}{\(%
+ \frac{1}{2} g_{\sigma\sigma}
\)}
\\
&
\colorbox{gray!50}{\(%
  + \frac{1}{6} g_{xxx} \left((x_{t-1} - \bar{x}) \otimes (x_{t-1} - \bar{x}) \otimes (x_{t-1} - \bar{x})\right)
  + \frac{1}{6} g_{uuu} \left(u_t \otimes u_t \otimes u_t\right)
\)}
\\
&
\qquad
\colorbox{gray!50}{\(%
+ \frac{3}{6} g_{xxu} \left((x_{t-1} - \bar{x}) \otimes (x_{t-1} - \bar{x}) \otimes u_t \right)
+ \frac{3}{6} g_{xuu} \left((x_{t-1} - \bar{x}) \otimes u_t \otimes u_t \right)
\)}
\\
&
\qquad\qquad
\colorbox{gray!50}{\(%
  + \frac{3}{6} g_{xx\sigma} \left((x_{t-1} - \bar{x}) \otimes (x_{t-1} - \bar{x}) \right)
  +             g_{xu\sigma} \left((x_{t-1} - \bar{x}) \otimes u_t \right)
+ \frac{3}{6} g_{uu\sigma} \left(u_t \otimes u_t \right)
\)}
\\
&
\qquad\qquad\qquad
\colorbox{gray!50}{\(%
  + \frac{3}{6} g_{x\sigma\sigma} (x_{t-1} - \bar{x})
  + \frac{3}{6} g_{u\sigma\sigma} u_t
\)}
\\
&
\qquad\qquad\qquad\qquad
\colorbox{gray!50}{\(
  + \frac{1}{6} g_{\sigma\sigma\sigma}
\)}
\end{align*}
where after running \texttt{stoch\_simul{(order=3)}}

\begin{itemize}
\item
\(x_t\) are the state variables and \(y_t\) are the variables that we focus our analysis on (i.e.\ reported or observed variables)

\item
the steady-states \(\bar{x}\) and \(\bar{y}\) can be accessed from the rows of\\
\texttt{oo\_.dr.ys{(oo\_.dr.order\_var)}}

\item
the \colorbox{gray!10}{first-order terms} can be accessed from the rows of\\
\texttt{oo\_.dr.ghx} and \texttt{oo\_.dr.ghu}

\item
the \colorbox{gray!25}{second-order terms} can be accessed from the rows of\\
\texttt{oo\_.dr.ghxx}, \texttt{oo\_.dr.ghxu}, \texttt{oo\_.dr.ghuu}, and \texttt{oo\_.dr.ghs2}

\item
the \colorbox{gray!25}{third-order terms} can be accessed from  the rows of \texttt{oo\_.dr.ghxxx},\\
\texttt{oo\_.dr.ghxxu}, \texttt{oo\_.dr.ghxuu}, \texttt{oo\_.dr.ghxss}, \texttt{oo\_.dr.ghuuu} and \texttt{oo\_.dr.ghuss}

\end{itemize}

Note that there is no \texttt{oo\_.dr.ghs}, \texttt{oo\_.dr.ghxxs}, \texttt{oo\_.dr.ghxus}, \texttt{oo\_.dr.ghuus}, and \texttt{oo\_.dr.ghs3}.

Why does Dynare not compute \(g_{\sigma}\), \(g_{xx\sigma}\), \(g_{xu\sigma}\), \(g_{uu\sigma}\), and \(g_{\sigma\sigma\sigma}\)?
\end{enumerate}

Now consider Table (2) of the paper.

\begin{enumerate}[resume]
\item
Replicate column (2) labeled \emph{Benchmark}. Note that this requires to run a simulation of length 2000000 for a third-order approximation to the model with Gaussian shocks.
In the hints below you will find out how to get the same shock series to replicate the numbers in the column.

\item
Replicate column (3) labeled \emph{Case I}. Note that this requires to run a simulation of length 2000000 for a third-order approximation to the model with non-symmetric shocks.
In the hints below you will find out how to get the same shock series to replicate the numbers in the column.

\item
Comment on the different results obtained in columns (2) and (3) of the table.
\end{enumerate}

\newpage

\subsection*{Codes and hints}

\begin{itemize}

\item
The mod file \texttt{Andreasen\_2012\_rare\_disasters.mod} contains a Dynare version of the DSGE model used to study rare disasters.
This mod file simulates data and computes empirical moments from a simulated time series of length 2000000 for a first-order approximation to the model.
Note that your task is to adapt the codes that do the simulation for a third-order approximation.
The mod file also contains some further hints and explanations.

\item
The function \texttt{Andreasen\_2012\_get\_shocks.m} produces both the Gaussian and non-symmetric shock series used to replicate columns (2) and (3) in Table 2 of the paper.
It also creates \(\Sigma^{(3)}\) for both cases which is an input argument of \texttt{get\_ghs3.m}.

\item
The function \texttt{get\_ghs3.m} computes \(g_{\sigma\sigma\sigma}\) from
\begin{align*}
(A+B) g_{\sigma\sigma\sigma} = -\left( f_{y^{**}_{+}} g^{**}_{uuu} + f_{y^{**}_{+}y^{**}_{+}} (g^{**}_u \otimes g^{**}_{uu})P^3_{u\_uu} + f_{y^{**}_{+}y^{**}_{+}y^{**}_{+}} (g^{**}_u \otimes g^{**}_u \otimes g^{**}_u)\right)\Sigma^{(3)}
\end{align*}
where \(f_{y^{**}_{+}}\), \(f_{y^{**}_{+}y^{**}_{+}}\), and \(f_{y^{**}_{+}y^{**}_{+}y^{**}_{+}}\) are the first, second, and third-order derivatives of the dynamic model equations with respect to forward jumper variables \(y_{t+1}^{**}\).
\(g^{**}_u\), \(g^{**}_{uu}\), and \(g^{**}_{uuu}\) are the partial derivatives of the jumper variables with respect to shocks.
\(P^3_{u\_uu}\) is a permutation matrix.
\(\Sigma^{(3)} = E[u_t \otimes u_t \otimes u_t]\) are the vectorized unconditional third-order product moments.
\(A\) and \(B\) are the auxiliary perturbation matrices:
\begin{align*}
A & = f_{y_0} + \begin{pmatrix} \underbrace{0}_{n\times n_{static}} &\vdots& \underbrace{f_{y^{**}_{+}} \cdot g^{**}_{x}}_{n \times n_{spred}} &\vdots& \underbrace{0}_{n\times n_{frwd}}  \end{pmatrix}\\
B & = \begin{pmatrix} \underbrace{0}_{n \times n_{static}}&\vdots & \underbrace{0}_{n \times n_{pred}} & \vdots & \underbrace{f_{y^{**}_{+}}}_{n \times n_{sfwrd}} \end{pmatrix}
\end{align*}
where \(f_{y_{0}}\) are the first derivatives of the dynamic model equations with respect to all current endogenous variables \(y_{t}^{}\)
and \(g^{**}_{x}\) are the first partial derivatives of the jumper variables with respect to previous state variables.

\item
In MATLAB, the Kronecker product \(\mathcal{A} \otimes \mathcal{B} \otimes \mathcal{C}\) of arbitrary matrices \(\mathcal{A}\), \(\mathcal{B}\), and \(\mathcal{C}\)
  can be computed using \texttt{kron{(\(\mathcal{A}\),kron{(\(\mathcal{B}\),\(\mathcal{C}\))})}}.

\item
If your computer struggles with the simulation, reduce the length of the simulation to 200000 or lower.
\end{itemize}

\newpage


\section[Calvo vs. Rotemberg]{Calvo vs. Rotemberg\label{ex:CalvoRotemberg}}
Consider a New Keynesian model, where the representative household maximizes lifetime utility
\begin{align*} 
E_t \sum_{j=0}^\infty \beta^j\left( \frac{c_{t+j}^{1-\sigma}}{1-\sigma} - \gamma \frac{n_{t+j}^{1+\phi}}{1+\phi} \right)
\end{align*}
subject to the period-by-period budget constraint
\begin{align*}
P_t c_t + \frac{1}{R_t} B_t = W_t n_t + Div_t + B_{t-1}
\end{align*}
where the notation of variables and parameters follows the lecture.
The following first-order conditions hold
\begin{align}
c_t^{-\sigma} &= \beta E_t \left[ \frac{R_t}{\pi_{t+1}} c_{t+1}^{-\sigma}  \right] \label{eq:CalvoRotemberg:Euler}
\\
w_t &= \gamma n_t^\phi c_t^\sigma \label{eq:CalvoRotemberg:LaborSupply}
\end{align}
where \(\pi_t = P_t/P_{t-1}\) and \(w_t = W_t/P_t\).
The final good market is competitive and the production function is given by the Dixit-Stiglitz aggregator
\begin{align*}
y_t = {\left(\int_0^1 {(y_t(f))}^{\frac{\epsilon-1}{\epsilon}}df\right)}^{\frac{\epsilon}{\epsilon-1}}
\end{align*}
Final good producers' demand for intermediate inputs is therefore equal to
\begin{align*}
y_t(f) = {\left(\frac{P_t(f)}{P_t}\right)}^{-\epsilon} y_t
\end{align*} 
Intermediate inputs \(y_t(f)\) are produced by a continuum of firms indexed by \(f\in[0,1]\) with the following linear technology:
\begin{align*}
y_t(f) = a_t n_t(f)
\end{align*}
where labor is the only input and the technology process \(a_t\) follows
\begin{align}
\ln(a_t) = \rho_a \ln(a_{t-1}) + \varepsilon_{a,t}, \qquad \varepsilon_{a,t} \sim N(0,\sigma_a^2) \label{eq:CalvoRotemberg:Technology}
\end{align}
Note that given the linear production function, real marginal costs \(mc_t(f)\) are the same across firms and equal to the productivity-adjusted real wage:
\begin{align}
mc_t(f) = mc_t = \frac{w_t}{a_t} \label{eq:CalvoRotemberg:MarginalCosts}
\end{align}
The intermediate-good sector is monopolistically competitive.
The price setting behavior follows either the Rotemberg (1982) or the Calvo (1983) specification as detailed below.

\paragraph{Rotemberg model}
The monopolistic firm faces a quadratic cost of adjusting nominal prices, which can be measured in terms of the final good:
\begin{align*}
\frac{\varphi}{2} {\left(\frac{P_t(f)}{P_{t-1}(f)}-1\right)}^2 y_t
\end{align*}
where \(\varphi>0\) determines the degree of nominal price rigidity.
Firms can change their price in each period subject to the payment of adjustment cost;
  hence, all firms face the same problem and will choose the same price and produce the same quantity:
  \(P_t(f) = P_t\) and \(y_t(f) = y_t\) for all \(f\).
Therefore, from the first-order condition, after imposing the symmetric equilibrium, we get:
\begin{align}
1- \varphi(\pi_t - 1) \pi_t + \varphi \beta E_t {\left(\frac{c_{t+1}}{c_t}\right)}^{-\sigma} \left[(\pi_{t+1}-1)\pi_{t+1} \frac{y_{t+1}}{y_t}\right]
  = (1-mc_t)\epsilon \label{eq:Rotemberg:PriceSetting}
\end{align}
As all the firms employ the same amount of labor, the aggregate production function is given by
\begin{align}
y_t = a_t n_t \label{eq:Rotemberg:AggregateSupply}
\end{align}
and the aggregate resource constraint takes the adjustment cost into account
\begin{align}
y_t = c_t + \frac{\varphi}{2} {\left(\pi_t-1\right)}^2 y_t \label{eq:Rotemberg:ResourceConstraint}
\end{align}


\paragraph{The Calvo model}
For each period there is a fixed probability \(1-\theta \) that a firm can re-optimize its nominal price \(P_t^*(f)\).
The joint dynamics of the optimal reset price \(p_t^* = P_t^*(f)/P_t\) are given by
\begin{align}
p_t^* &= \frac{\epsilon}{\epsilon-1} \frac{q^1_t}{q^2_t} \label{eq:Calvo:OptimalPrice}
\\
q^1_t &= mc_t y_t^{1-\sigma} + \theta \beta E_t \left[\pi_{t+1}^\epsilon q^1_{t+1}\right] \label{eq:Calvo:OptimalPriceAux1}
\\
q^2_t &= y_t^{1-\sigma} + \theta \beta E_t \left[\pi_{t+1}^{\epsilon-1} q^2_{t+1}\right] \label{eq:Calvo:OptimalPriceAux2}
\end{align}
where we used the resource constraint:
\begin{align}
y_t = c_t \label{eq:Calvo:ResourceConstraint}
\end{align}
The evolution of the aggregate price index implies
\begin{align}
p_t^* &= {\left(\frac{1-\theta \pi_t^{\epsilon-1}}{1-\theta}\right)}^{\frac{1}{1-\epsilon}} \label{eq:Calvo:AggregatePriceIndex}
\end{align}
Aggregate output can be expressed as
\begin{align}
y_t = \frac{a_t}{s_t} n_t \label{eq:Calvo:AggregateSupply}
\end{align}
where \(s_t\) denotes a measure of price dispersion and evolves according to
\begin{align}
s_t = \int_0^1 {\left(\frac{P_t(f)}{P_t}\right)}^{-\epsilon}df = (1-\theta){(p_t^*)}^{-\epsilon} + \theta \pi_t^\epsilon s_{t-1} \label{eq:Calvo:PriceDispersion}
\end{align}

\paragraph{Monetary Policy}
Monetary policy follows a Taylor rule:
\begin{align}
\ln\left(\frac{R_t}{R}\right) = \alpha_\pi \left( \ln(\pi_t) - \ln\left({\left(1+\frac{\bar{\pi}^{A}}{100}\right)}^{1/4}\right) \right) + \alpha_y \ln\left(\frac{y_t}{y}\right) \label{eq:CalvoRotemberg:TaylorRule}
\end{align}
where \(R\) and \(y\) are steady-state values and \(\bar{\pi}^{A}\) the annual inflation target of the central bank.

\paragraph{Parametrization}
If not otherwise stated, use the following parameter values for the exercises:
\begin{align*}
\beta &= 0.99,\quad \sigma=1 \quad \phi=1, \quad \epsilon = 10, \quad \rho_a = 0.9, \quad \sigma_a = 0.01, \quad \alpha_\pi = 1.5
\\
\alpha_y &= 0.125,\quad \theta = 0.75, \quad \varphi = \frac{(\epsilon-1)\theta}{(1-\theta)(1-\beta\theta)}, \quad \gamma = 8.1, \quad \bar{\pi}^{A} = 0 
\end{align*}

\subsection*{Hints}
\begin{itemize}
\item The Rotemberg model is given by equations%
~\eqref{eq:CalvoRotemberg:Euler},~\eqref{eq:CalvoRotemberg:LaborSupply},~\eqref{eq:CalvoRotemberg:Technology},~\eqref{eq:CalvoRotemberg:MarginalCosts},%
~\eqref{eq:Rotemberg:PriceSetting},~\eqref{eq:Rotemberg:AggregateSupply},~\eqref{eq:Rotemberg:ResourceConstraint}, and~\eqref{eq:CalvoRotemberg:TaylorRule}.
A Dynare implementation is given in the file \texttt{Rotemberg.mod}.
\item The Calvo model is given by equations%
~\eqref{eq:CalvoRotemberg:Euler},~\eqref{eq:CalvoRotemberg:LaborSupply},~\eqref{eq:CalvoRotemberg:Technology},~\eqref{eq:CalvoRotemberg:MarginalCosts},%
~\eqref{eq:Calvo:OptimalPrice},~\eqref{eq:Calvo:OptimalPriceAux1},~\eqref{eq:Calvo:OptimalPriceAux2},~\eqref{eq:Calvo:ResourceConstraint},%
~\eqref{eq:Calvo:AggregatePriceIndex},~\eqref{eq:Calvo:AggregateSupply},~\eqref{eq:Calvo:PriceDispersion}, and~\eqref{eq:CalvoRotemberg:TaylorRule}.
A Dynare implementation is given in the file \texttt{Calvo.mod}.
\item Hat variables are defined as log-deviations from the steady state, e.g.
\(\hat{y}_t = \ln(y_t) - \ln(y)\).
\end{itemize}

\newpage

\subsection*{Exercises}

\begin{enumerate}

\item Assume a zero inflation target, \(\bar{\pi}^{A}=0\), such that steady-state inflation is one, \(\pi = 1\).
Log-linearize the labor supply equation (\ref{eq:CalvoRotemberg:LaborSupply}),
  the marginal costs equation (\ref{eq:CalvoRotemberg:MarginalCosts}),
  the Rotemberg aggregate supply equation (\ref{eq:Rotemberg:AggregateSupply}),
  and the Rotemberg resource constraint (\ref{eq:Rotemberg:ResourceConstraint}).
Combine these to get the following log-linearized equation for marginal costs:
\begin{align}
\hat{mc}_t = (\phi+\sigma) \hat{y}_t - (1+\phi) \hat{a}_t \label{eq:LogLinearization.MarginalCosts}
\end{align}

\item Assume a zero inflation target, \(\bar{\pi}^{A}=0\), such that steady-state inflation is one, \(\pi = 1\).
Derive the New Keynesian IS curve
\begin{align*}
\hat{y}_t = E_{t} \hat{y}_{t+1} - \frac{1}{\sigma} \left( \hat{R}_t - E_{t} \hat{\pi}_{t+1} \right)
\end{align*}
by log-linearizing the Euler equation~\eqref{eq:CalvoRotemberg:Euler}
and the Rotemberg resource constraint~\eqref{eq:Rotemberg:ResourceConstraint}.

\item Assume a zero inflation target, \(\bar{\pi}^{A}=0\), such that steady-state inflation is one, \(\pi = 1\).
Derive the New Keynesian Phillips curve:
\begin{align*}
\hat{\pi}_t = \beta E_{t} \hat{\pi}_{t+1} + \frac{\varepsilon-1}{\varphi} \hat{mc}_t
\end{align*}
by log-linearizing the Rotemberg price setting equation~\eqref{eq:Rotemberg:PriceSetting}.

\item Use Dynare to show that a zero inflation target \((\bar{\pi}^{A}=0)\) implies observational equivalence between the Rotemberg and Calvo model.

\item Investigate the steady-state properties of the two models when the targeted inflation rate of the central bank goes up, \(\bar{\pi}^{A} = \{1, 2, 3, 4, 5\} \).

\begin{enumerate}
  \item What happens to the steady-state values in both models when compared to the zero-inflation steady-state values?
  Are the models still observational equivalent with respect to the steady-state?
  \item What can you say about the slope of the long-run Phillips Curve, i.e.\ what happens to steady-state output if the inflation target goes up?
\end{enumerate}

\item Now compare the impulse-response functions of output and inflation to a one standard deviation TFP shock of the two price-setting models
  when the inflation target of the central bank changes \(\bar{\pi}^{A} = \{1, 2, 3, 4, 5\} \).
Are the models still observational equivalent?
Provide intuition for the results.

\item Now compute the determinacy region for \(\phi_\pi \in [0,6]\) and \(\phi_y \in [-1,6]\) for both model variants
  when the inflation target goes up \(\bar{\pi}^{A} = \{1, 3, 5\} \).
What does this imply for the ability of monetary policy to \emph{anchor inflation expectations}.
Hint: Use Dynare's sensitivity toolbox.

\item On July 8, 2021 the ECB Governing Council announced a new monetary policy strategy
  that adopts a symmetric 2\% inflation target over the medium term instead of previously aiming at ``below but close to 2\%'' target.
Given the insights you gained in this exercise:
Would it be desirable for policymakers to target a higher long-run rate of inflation?
How does the zero-lower-bound (ZLB) affect this decision?

\end{enumerate}

\newpage

\section[Disaster Risk and Preference Shifts]{Disaster Risk and Preference Shifts in a New Keynesian Model\label{ex:DisasterRisk}}
Read the paper by \emph{Isoré and Szczerbowicz (2017, JEDC)} on ``Disaster risk and preference shifts in a New Keynesian model''.

\subsection*{Exercises}

\begin{enumerate}
\item What is the main research question addressed in this paper?
How does it relate to the existing literature on disaster risk, particularly Gourio (2012)?

\item Describe the model framework used by the authors.
What are the key departures from a standard New Keynesian model?
What is the ``puzzle'' that the authors identify in RBC models with disaster risk?
How does their New Keynesian framework resolve this puzzle?
Discuss the role of:
\begin{itemize}
\item The elasticity of intertemporal substitution (EIS)
\item Price stickiness
\end{itemize}

\item According to the paper, what constitutes a ``rare disaster''?
Provide the formal definition using the indicator variable \(x_{t+1}\) and probability \(\theta_t\).
Explain the crucial distinction between (i) an actual disaster occuring and (ii) a change in disaster probability.

\item Gourio's stationarization trick is a key technical innovation of the paper
  to eliminate the binary disaster variable \(x_{t+1}\) from the equilibrium system.
Illustrate this with the capital accumulation equation:
\begin{equation*}
K_{t+1} = \left[(1-\delta_t)K_t + S\left(\frac{I_t}{K_t}\right)K_t\right]e^{x_{t+1}\ln(1-\Delta)}
\end{equation*}
Show how to transform this equation using detrended capital \(k_t \equiv K_t/z_t\)
  and investment \(i_t \equiv I_t/z_t\) to eliminate \(x_{t+1}\)
  such that the large disaster event variable is no longer present in the equilibrium system,
  but only the small disaster risk shock remains.
Why is it crucial that both capital destruction and productivity growth slowdown are affected by the same destruction rate \(\Delta \)?

\item The paper argues that despite disasters being large events (with destruction rate \(\Delta \approx 0.22\)),
  third-order perturbation methods remain valid.
Comment on the following:
  \begin{itemize}
  \item Why would one typically expect perturbation methods to fail for large shocks?
  \item How does the paper's approach circumvent this problem? (Hint: What is actually being perturbed?)
  \item Why is a third-order approximation necessary for the analysis?
  In other words, what would be missing with only first- or second-order approximations?
  \end{itemize}

\end{enumerate}

\end{document}